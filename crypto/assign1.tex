\documentclass[10pt,letterpaper]{article}
\usepackage[latin1]{inputenc}
\usepackage{amsmath}
\usepackage{amsfonts}
\usepackage{amssymb}
\usepackage{graphicx}
\usepackage{seqsplit}
\title{Assignment 1}
\author{Andrew Rosen}
\begin{document}
\maketitle
\section*{Question 0 }
There are a couple of ways this could work.  If we see substitution of Chinese characters for other Chinese characters, we are looking at a frequency table with an enormous $x$ axis.  This just means we need a larger corpus to use brute force method.

On the other hand, Chinese is written down phonetically, it would be a frequency analysis on syllables, which is also a larger space.

\section{}


\section{}
No, there aren't enough letters.

\section{}
We can discount other cipher systems because it's unlikely that an othery cipher system would yield a ciphertext consisting only of \texttt{adfgvx}.

It's almost certainly permuted, as $\Phi = 0.00539088589576$ on the corpus broken up into pairs of symbols, which looks just like random text.  If it wasn't permuted,$\Phi$ would be much closer to 0.0385.
\newpage
\section{}
A $\Phi$ of 0.0363075549455 was found with 115 columns. This was significantly larger than the other $\Phi$ values.  The resulting text:


\texttt{\seqsplit{vxvdaagvaffvxvdddddxavafvggvvdaagvvxaadfvggggvggvgfdxvdxgaaavggvdxxfxfddavggdgdgdfdxaaavxfafdxfvgvafvgddaavggadfafvdaaaxdxfadxdxvgaadgaaavgvdfavgvggafgvaafvaafaggdfgvvddxvaaffvdxaafvdxvdaaavvaxfdfdxaxdxavxfdxddgvdgdxddggavaavaaadfgvvddxdfvaafvavggggvfxvgggvxgvvddxvggvvdaagvafgvvxaafvdgdfxfafaxdxavdgdxvaafgaaaxffvgaafdxvggadxvxaafvafvgaaddaaavxfdxfvfvaavaaxaavggadxvafvgvaagvdxgvvdaavgvgggvxaavgvagvvddxdfxvfvdxvagvggfgxvgvafgvvaggvxvggvggxfaaxaafvgdxfvfvvxvddfdfggxvfvfxxvxffxafvgfdxfafgvgvxfdxvadxaxafxfdfggxvgvvddxdfvxggxvxfvafvaadffddxgvxvfgaavgvavaggfvggdgdxgvvdafvgfdddggavdfggxvavxfafaxafvgfdgaaavggvdfggxvvggggvfxvgggvxafvgfdggddgaggxvavfvdxgvvdaagvafvxaafvafxfxfaavgvagvvddxdfvaafvavggvfdafaxdxdgdxfgafxfxffvgvvddxdffdaaaxdxdgdxgaxfxvdgfgfvggvggvvddxfvafvadxggddgvvddxvddxaavaaavgvafvgvavaavgfddxaafvafgvdgaadfaafgfgdxaaavgvvdggfvdxgaxfxvdgfgfvggvggvvddxvddxaavaggddgvdxvggaxvavdxvadgdxddggavgvvddxgvafdgdxfadxafvgfdafvdaaaxdxfxvgggvxvgggvgdxgaxfxvdgfgggvggvvddxvddxaavavdaaaxdxdgggavdxdxdddddxgagvxvfgggvgdgdfxfafaxdxavaavgvadgaafxdxdgdxdddxdxxfdgggavdxaavgffafggxvfvgvggfdggfvgvavaaaffdvdgvaavxaadfgvvddxvgaavgvagvvddxavdxaavgvavaggvxvdaagvvxaafvvxaavggvdxvagvggfadxvaggvgdxvxafgvvdggxvgvddxvavgvvddxavxfggfvfvggddgvafdgdxgvvdaavgaavxvdggxfdxfaggffggddfgafxfxffvvaggdxfvvgggvxdfggxvfxvgggvxafgvggddgvdxvgaffvfvgggvvdggfvdxfvafdgfgxfdxggxfvaddaafvvdafggvgdxvaavdxdgdxvaafdxfvaaavdxfvggdgdxgvafdgdxfvdgggavdxdxddddafgaaagaafggxvfvgvvdaavgaaxfxfgvvddxvaaffvfgdxvgfvaaavdffvgvxvdddd}
}

\newpage
\section{}
I decided to make a crib of the first 10 occurrences of the symbol \texttt{gg}, which occur at indices \texttt{[21, 23, 37, 73, 80, 109, 123, 127, 194, 208]}.

I searched each substring of the \texttt{3boat10} as the same length as the corpus, looking for an instance where characters at the listed indices were identical.  The found substring was:

\texttt{\seqsplit{whatisufferinthatwaynotonguecantellfrommyearliestinfancyihavebeenamartyrtoitasaboythediseasehardlyeverleftmeforadaytheydidnotknowthenthatitwasmylivermedicalsciencewasinafarlessadvancedstatethannowandtheyusedtoputitdowntolazinesswhyyouskulkinglittledevilyoutheywouldsaygetupanddosomethingforyourlivingcantyounotknowingofcoursethatiwasillandtheydidntgivemepillstheygavemeclumpsonthesideoftheheadandstrangeasitmayappearthoseclumpsontheheadoftencuredmeforthetimebeingihaveknownoneclumpontheheadhavemoreeffectuponmyliverandmakemefeelmoreanxioustogostraightawaythenandthereanddowhatwaswantedtobedonewithoutfurtherlossoftimethanawholeboxofpillsdoesnowyouknowitoftenissothosesimpleoldfashionedremediesaresometimesmoreefficaciousthanallthedispensarystuffw}
}\end{document}