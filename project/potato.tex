\documentclass[10pt,a4paper]{article}

\linespread{1.3}
\usepackage[utf8]{inputenc}
\usepackage{amsmath}
\usepackage{amsfonts}
\usepackage{amssymb}
\author{Andrew Rosen}
\title{Analytical and Practical Evaluation of Sybil Attacks}
\begin{document}
\maketitle


\section{Introductions}
Security analysis typically assumes an omnicient attacker. 

I want to practically demonstrate this as well demonstrating how easy it is to place nodes in regions.

Most DHTs are vulnerable to a \textit{sybil attack} or \textit{eclipse attack}.


This is a lot like a birthday attack, only searching for a collision with a region


Why am I doing this?
DHTS are important cause I like them.

Security is not something that is thought about for a DHT, unless the 
DHT is specifically made to be secure against X.  
Or it's left to the applications

\section{Analysis}


\section{Simulations}

Experiment 1: Potatoing 2 random nodes
The amount of time to potato two random hashkeys was 29.6218323708 microseconds, and was achevied $ 99.996\%$ of the time



Experiment 2:  Naive Complete Eclipse via Sybil
Attack objective is to place

\end{document}