\documentclass[10pt,a4paper]{article}
\linespread{1.3}
\usepackage[utf8]{inputenc}
\usepackage{amsmath}
\usepackage{amsfonts}
\usepackage{amssymb}
\author{Andrew Rosen}
\title{Analytical and Practical Evaluation of Sybil Attacks}
\begin{document}
\maketitle



%TODO IN THIS ORDER
%1) WRITE PAPER
%2) DO EXPERIMENTS FOR GRAPHS
%3) ADD EXPERIMENT RESULTS TO PAPER
%4) ADD REFERENCES
%5) MAKE UP SOME SLIDES
\begin{abstract}
This paper explores the feasibility of performing naive Sybil attacks that completely occlude healthy nodes from each other.
The vulnerability of Distributed Hash Tables to Sybil attacks and Eclipse attacks has been well known for some time.
However, these vulnerabilities has often been explored in a theoretical sense, assuming the attacker is a global adversary from the beginning, nigh-omniscient and omnipotent. 
This paper seeks from a analytical and practical perspective, how valid that assumption is.

We examine the amount of computational effort required to become a global adversary starting as a non-global adversary.
We do this by analyzing the amount of time it takes an attacker with a given IP to choose a port to obtain a desired hashkey, a process we call \emph{potatoing}.
We present potatoing to emphasize the ease of this attack, but also demonstrate potential non-security uses of potatoing that are beneficial to DHT load-balancing. 
\end{abstract}

\section{Introduction}
One of the key properties of structured peer-to-peer (P2P) systems is the lack of a centralized coordinator.
P2P systems remove the vulnerability of a single point of failure and and the suceptability to a denial of service attack (cite original Sybil paper), but in doing so, open themselves up to new attacks.
Completely decentralized P2P systems are vulnerable to \textit{Eclipse attacks}, whereby an attacker completely occlude healthy nodes from one another.
This prevents them from communicating without being intercepted by the attacker.

One way to accomplish this attack is to perform a \texttt{Sybil Attack}.
In a Sybil attack, the attacker creates multiple malicious virtual nodes in order to disrupt the network.
If enough malicious nodes are injected into the system, the majority of the nodes will be occluded from one another, successfully performing an Eclipse attack.


Security analyses typically assumes an adversary using a Sybil is omniscient and can inject virtual nodes wherever he chooses in a reasonable amount of time. 
Our goal is to demonstrate how veracity of that assumption by doing analysis and simulations.

%Why am I doing this?
Sybil attacks represent a significant threat to the security of any distributed system.
Many of the analysises on Tor emphasize the vulnerability of Tor to the Sybil attack.
P2P systems like BitTorrent are essential , and current research demonstrates that they it is vulnerable to this attack.
There have been many suggestions on how to defend against Sybil attack, but the only surefire way is to introduce a trusted authority for identities.
This solution potentially removes the Sybil attack, but reintroduces vulnerabilities to denial of service attacks, bringing us full circle.


%This is a lot like a birthday attack, only searching for a collision with a region


Our research has very broad implications 
DHTS are important cause I like them.
Hashing 

Any application built using a DHT must be address its vulnerabilities to the Eclipse and Sybil attacks

Security is not something that is thought about for a DHT, unless the 
DHT is specifically made to be secure against X.  
Or it's left to the applications


A complete DHT occlusion is overkill.


%\begin{itemize}
%    \item We first discuss the mechanics of performing a Sybil attack and analyze it's effectiveness from a analytical standpoint.
%    \item Our experiments demonstrate the trivial amount of time and computational effort needed to
%\end{itemize}

\section{Analysis}

\cite{bellare2004hash}

The birthday attack analysis says given so many elements, likelyhood of collision between any of these elements.
The potato attack says given this region, what is likelyhood i can find something in this region.  
It's a different analysis since I'm looking for 1 attacker colliding with one specified region at a time.

Suppose we have a DHT with $N$ members in it, with the hashspace of $[0,2^{160})$.
The case of small $N$ is ignored, since they are trivial even when unbalanced.
We can assume that, for a large enough $N$, node IDs will beclose to evenly distributed accross the network, meaning there will be $\approx \frac{2^{160}}{N}$ hashkeys between each node ID.


Size of bin is $$ bin =  \frac{H}{N} $$

The probability $P$ of an attacker finding an hashkey that lands in the range with the range $(n,m)$ is 
$$ P \approx \frac{n-m}{H}\cdot num\_ips \cdot num\_ports  $$

making it equivalent to
$$ P \approx \frac{1}{N}\cdot num\_ips \cdot num\_ports  $$


The chances of compromising an entire network of N nodes that partition the entire network.
$$c \approx  1 - (1 -\frac{1}{N})^{P*I})  $$

Alternatively, we can view this as doing a birthday attack in progress with different probabilities.
EG, we've generated $\frac{h}{N}$ values already, how many more do we need?


\section{Simulations}
Simulations were performed on a computer with consumer-grade budget hardware. 


\subsection{Experiment 1:Potatoing 2 random nodes}
Our initial experiment was designed to establish the feasability of injecting in between two random nodes.

Each trial, we generated two victims with random IP/port combinations, and an attacker with a random IP.
The experiment was for the attacker to find a hashkey in between the two victim's key, from the lowest to highest.

The amount of time to potato two random hashkeys was 29.6218323708 microseconds, and was achevied $ 99.996\%$ of the time.



\subsection{Experiment 2:  Nearest Neighbor Eclipse via Sybil}
The objective of the second experiment is to completely ensnare a network using a Sybil attack, starting with single node.


\subsection{Experiment 3: Fully Complete Eclipse via Sybil}
The previous experiments of

In some of the below frameworks, it is more efficient to perform an Eclipse attack by falsely advertising, rather than injecting


\subsubsection{Chord}
Most of the above attacks



Chances the finger is already covered:

\subsubsection{Kademlia}

\subsubsection{Plaxton Based networks}
More efficient to lie.


\section{Masking the Attack}
Now that we have established that a Sybil attack can be performed with great ease, our focus now turns to avoiding detection.

We need a different IP for each point surrounding our victim.  In the Nearest-Neighbor attack, we need a 

We can reduce this into an interesting graph coloring problem

\section{Simple Load Balancing Injection Framework}

Costs:  Need to hold 15000ish hashkeys, effectily 160-bit numbers

\bibliography{potato}
\bibliographystyle{plain}
\end{document}
