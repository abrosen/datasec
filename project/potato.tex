\documentclass[10pt,a4paper]{article}

\linespread{1.3}
\usepackage[utf8]{inputenc}
\usepackage{amsmath}
\usepackage{amsfonts}
\usepackage{amssymb}
\author{Andrew Rosen}
\title{Analytical and Practical Evaluation of Sybil Attacks}
\begin{document}
\maketitle

\begin{abstract}
This paper explores the feasiblity of performing naive Sybil attacks that completely occlude healthy nodes from each other.
The vulnerability of Distributed Hash Tables to Sybil attacks and Eclipse attacks has been well known for some time.
However, these vulnerabilities has often been explored in a theoretical sense, assuming the attacker to be nigh-omnicient and omnipotent. 
This paper seeks from a analytical and practical perspective, how valid that assumption is.


We do this by analyzing the amount of time it takes an attacker with a given IP to choose a port to obtain a desired hashkey, a process we call \emph{potatoing}.
We present potatoing to emphasize the ease of this attack, but also demonstrate potential non-security uses of potatoing that are benificial to DHT load-balancing. 
\end{abstract}

\section{Introductions}
Security analysis typically assumes an omnicient attacker. 

I want to practically demonstrate this as well demonstrating how easy it is to place nodes in regions.

Most DHTs are vulnerable to a \textit{sybil attack} or \textit{eclipse attack}.


This is a lot like a birthday attack, only searching for a collision with a region


Why am I doing this?
DHTS are important cause I like them.

Security is not something that is thought about for a DHT, unless the 
DHT is specifically made to be secure against X.  
Or it's left to the applications


A complete DHT occlusion is overkill.
\section{Analysis}

The birthday attack analysis says given so many elements, likelyhood of collision between any of these elements.
The potato attack says given this region, what is likelyhood i can find something in this region.  
It's a different analysis since I'm looking for 1 attacker colliding with one specified region at a time 


\section{Simulations}

Experiment 1: Potatoing 2 random nodes
The amount of time to potato two random hashkeys was 29.6218323708 microseconds, and was achevied $ 99.996\%$ of the time



Experiment 2:  Naive Complete Eclipse via Sybil
Attack objective is to place

\end{document}
